\documentclass[11pt]{article}
\usepackage[a4paper,margin=1in]{geometry}
\usepackage{amsmath,amssymb,amsfonts}
\usepackage{bm}
\usepackage{physics}
\usepackage{hyperref}

\title{Draft TerminalBench Science Task (Crystallography)\\
Powder Neutron Diffraction from Stacking-Disordered Ice with Reichweite-4 Disorder and Lorentzian Instrument Broadening}
\author{}
\date{}

\begin{document}
\maketitle

\section*{(1) Draft Question}

\subsection*{Problem overview}
You are given synthetic \emph{powder} neutron diffraction data from stacking-disordered ice (ice I$_{\mathrm{sd}}$), generated by:
\begin{enumerate}
\item a finite crystallite of size $(N_a,N_b,N_c)$ in the three lattice directions,
\item a stacking-disorder model with \emph{reichweite} $s=4$ described via ABC and HK notation and four transition parameters $(\alpha,\beta,\gamma,\delta)$,
\item an oxygen-only scattering model (hydrogen is ignored),
\item an instrument resolution function modeled as a global Lorentzian convolution in $Q=\|\bm Q\|$ with width parameter $\Gamma>0$,
\item additive noise (details provided in the input).
\end{enumerate}
Your goal is to implement a forward model and then solve the inverse problem of estimating
\[
\hat\alpha,\hat\beta,\hat\gamma,\hat\delta,\hat\Gamma
\]
from the observed powder pattern.

\subsection*{Scattering model from first principles}
We model \emph{only the oxygen sublattice}. Let $\mathcal{J}$ denote the set of all oxygen atoms in the crystallite. For a given scattering vector $\bm Q\in\mathbb{R}^3$, define the total crystal amplitude
\begin{equation}
\psi(\bm Q) \;=\; \sum_{j\in\mathcal{J}} f_j \, e^{-i\,\bm Q\cdot \bm r_j},
\label{eq:psi_def}
\end{equation}
where $\bm r_j$ is the position of oxygen atom $j$ and $f_j=f_{\mathrm O}$ is the (coherent) neutron scattering length of oxygen (provided; treat $f_{\mathrm O}$ as $Q$-independent and ignore Debye--Waller factors). Hydrogen is ignored in this model.

Under the kinematic approximation, the (un-normalized) differential cross section is
\begin{equation}
\frac{d\sigma}{d\Omega}(\bm Q)\;\propto\; \mathbb{E}\big[\,|\psi(\bm Q)|^2\,\big],
\label{eq:cross_section}
\end{equation}
where $\mathbb{E}[\cdot]$ is the expectation over stacking disorder (defined below). You may absorb proportionality constants into a global scale factor.

\subsection*{Crystal geometry and atomic positions}
You are given lattice vectors $\bm a,\bm b,\bm c$ and the three close-packed in-plane offsets $\bm t_A,\bm t_B,\bm t_C$ for the oxygen sublattice.

A finite crystallite is formed by:
\begin{itemize}
\item placing one oxygen per in-plane lattice site on an $N_a\times N_b$ grid in the $(\bm a,\bm b)$ directions, and
\item stacking $N_c$ close-packed layers along $\bm c$ with interlayer spacing $d$ (provided).
\end{itemize}

Each close-packed layer has a label $X_n\in\{A,B,C\}$ and is placed using an in-plane offset $\bm t_{X_n}$ (provided vectors $\bm t_A,\bm t_B,\bm t_C$). The atom positions are thus
\begin{equation}
\bm r_{n,m_1,m_2}
= m_1\bm a + m_2\bm b + n\,d\,\hat{\bm c}
+ \bm t_{X_n},
\quad
\begin{matrix}
0\le n\le N_c-1\\
0\le m_1\le N_a-1\\
0\le m_2\le N_b-1\\
\end{matrix}
\label{eq:positions}
\end{equation}
where $\hat{\bm c}$ is the unit vector parallel to $\bm c$.

\subsection*{Stacking disorder: ABC and HK notation (reichweite $s=4$)}
The stacking sequence $(X_n)_{n=0}^{N_c-1}$ obeys the close-packed constraint: $X_{n+1}\neq X_n$.

Define the Jagodzinski (HK) symbol $S_n\in\{H,K\}$ for the \emph{middle} layer of each triple:
\begin{equation}
S_n =
\begin{cases}
H, & X_{n-1} = X_{n+1},\\
K, & X_{n-1} \neq X_{n+1},
\end{cases}
\qquad (1\le n\le N_c-2).
\label{eq:HK_def}
\end{equation}

Assume $(S_n)$ is a stationary \emph{second-order} Markov chain (reichweite $s=4$) with transition probabilities
\begin{equation}
\alpha=P(S_{n}=K\mid S_{n-1}S_{n-2}=HH),\;
\beta =P(S_{n}=K\mid S_{n-1}S_{n-2}=HK),\;
\gamma=P(S_{n}=K\mid S_{n-1}S_{n-2}=KH),\;
\delta=P(S_{n}=K\mid S_{n-1}S_{n-2}=KK),
\label{eq:abcd_def}
\end{equation}
and $P(S_n=H\mid\cdot)=1-P(S_n=K\mid\cdot)$.

The joint law of the ABC sequence is the one induced by \eqref{eq:HK_def} and the Markov law \eqref{eq:abcd_def}.

\subsection*{Powder averaging}
For a scalar $Q>0$, the powder intensity is the orientational average over the sphere $\partial B_Q=\{\bm Q:\|\bm Q\|=Q\}$:
\begin{equation}
I(Q)\;=\;\frac{1}{4\pi Q^2}\int_{\partial B_Q}\frac{d\sigma}{d\Omega}(\bm Q)\,dS(\bm Q).
\label{eq:powder}
\end{equation}

\subsection*{Instrument resolution: Lorentzian in $Q$}
The observed powder pattern is modeled as a convolution in scalar $Q$:
\begin{equation}
I_{\mathrm{obs}}(Q)=\int_{0}^\infty I(Q')\,R_\Gamma(Q-Q')\,dQ',
\qquad
R_\Gamma(\Delta Q)=\frac{1}{\pi}\frac{\Gamma}{\Gamma^2+(\Delta Q)^2},\ \Gamma>0.
\label{eq:lorentz_conv}
\end{equation}
The dataset provides values $\{(Q_i, I_{\mathrm{obs}i})\}_{i=1}^M$ on a given $Q$-grid.

\subsection*{Task}
 Estimate $(\alpha,\beta,\gamma,\delta,\Gamma)$ from $\{(Q_i,I_{\mathrm{obs}i})\}$, subject to
\[
\alpha,\beta,\gamma,\delta\in[0,1],\qquad \Gamma>0.
\]

\subsection*{Output}
Return a JSON-like file containing your estimates $(\hat\alpha,\hat\beta,\hat\gamma,\hat\delta,\hat\Gamma)$.

\subsection*{Notes}

If the opimisation problem is too hard, consider supplying the agent multiple datasets with the same $\alpha,\beta,\gamma,\delta,\Gamma$. We could also fix some of these parameters (e.g $\Gamma$ and/or provide a good initial guess for the parameters)

\bigskip
\hrule
\bigskip

\section*{(2) Draft Solution (Techniques and Derivations; no code)}

\subsection*{A. Expand the intensity and separate geometry from disorder}
Start from
\[
|\psi(\bm Q)|^2 = \psi(\bm Q)\,\psi(\bm Q)^\ast
= \sum_{j\in\mathcal{J}}\sum_{j'\in\mathcal{J}} f_j f_{j'}\,e^{-i\bm Q\cdot(\bm r_j-\bm r_{j'})}.
\]
Taking expectation over stacking disorder gives
\begin{equation}
\mathbb{E}[|\psi(\bm Q)|^2]
= \sum_{j,j'\in\mathcal{J}} f_j f_{j'}\,\mathbb{E}\!\left[e^{-i\bm Q\cdot(\bm r_j-\bm r_{j'})}\right].
\label{eq:double_sum}
\end{equation}
Using the explicit indexing \eqref{eq:positions}, group atoms by \emph{layer index} $n$ and in-plane indices $(m_1,m_2)$. This yields a natural factorization:
\begin{itemize}
\item deterministic in-plane lattice sums over $m_1,m_2$, producing Dirichlet-kernel-type factors in the reciprocal components parallel to $\bm a^\ast,\bm b^\ast$,
\item a stacking-direction sum over layer indices $n$ that couples to the random offsets $\bm t_{X_n}$ (hence to the ABC process).
\end{itemize}

A convenient intermediate object is the \emph{single-layer amplitude} for a given layer label $X\in\{A,B,C\}$:
\begin{equation}
g_X(\bm Q) := f_{\mathrm O}\, e^{-i\bm Q\cdot\bm t_X}.
\label{eq:gX}
\end{equation}
Then $\psi(\bm Q)$ can be rewritten as
\begin{equation}
\psi(\bm Q)
=
\Big(\sum_{m_1=0}^{N_a-1}e^{-i\bm Q\cdot(m_1\bm a)}\Big)
\Big(\sum_{m_2=0}^{N_b-1}e^{-i\bm Q\cdot(m_2\bm b)}\Big)
\sum_{n=0}^{N_c-1} e^{-i\bm Q\cdot(n d\,\hat{\bm c})}\, g_{X_n}(\bm Q).
\label{eq:psi_factored}
\end{equation}
Squaring and taking expectation reduces the disorder dependence to correlations of $g_{X_n}$ across layer separations.

\subsection*{B. Reduce to a 1D correlation function along the stacking direction}
Let $\Delta=n'-n$. From \eqref{eq:psi_factored} and \eqref{eq:double_sum}, one obtains
\begin{equation}
\mathbb{E}[|\psi(\bm Q)|^2]
=
|S_{ab}(\bm Q)|^2
\sum_{\Delta=-(N_c-1)}^{N_c-1} (N_c-|\Delta|)\,
C_\Delta(\bm Q)\,e^{-i\bm Q\cdot(\Delta d\,\hat{\bm c})},
\label{eq:master_corr}
\end{equation}
where the deterministic in-plane factor is
\[
S_{ab}(\bm Q)=
\sum_{m_1=0}^{N_a-1}e^{-i\bm Q\cdot(m_1\bm a)}
\sum_{m_2=0}^{N_b-1}e^{-i\bm Q\cdot(m_2\bm b)},
\]
and the \emph{stacking correlation function} is (note the conjugation convention)
\begin{equation}
C_\Delta(\bm Q) := \mathbb{E}\!\left[g_{X_n}(\bm Q)^\ast\,g_{X_{n+\Delta}}(\bm Q)\right],
\qquad C_{-\Delta}=C_\Delta^\ast.
\label{eq:Cdef}
\end{equation}
Thus the forward problem becomes: compute $C_\Delta(\bm Q)$ efficiently for many $\Delta$ and many $\bm Q$.

\section{Closed-form expression in terms of transition matrices}
\label{sec:Cdelta_closed_form}

This derivation closely follows the matrix/Markov formulation in Hart--Hansen--Kuhs (2021). In particular, it parallels their treatment of the ``average structure factor product'' $Y_{m_3}(\bm Q)$ in terms of (i) a block Markov chain with transition matrix $\Xi$, (ii) the pair-correlation $G_{ij}(m_3)$, and (iii) a Hermitian ``structure matrix'' $F$.%
\footnote{See their Eq.~(9) (definition of $Y_{m_3}$ via $G_{ij}$ and block structure factors) and Appendix~B where $G_{ij}(m_3)=\pi_i(\Xi^{m_3})_{ij}$ and $Y_{m_3}$ is written as a trace.}%
\cite{hart2021acentric}

\subsection{From layer stacking to a first-order Markov chain on a finite state space}
Let $(Z_n)_{n\in\mathbb{Z}}$ be a \emph{finite-state} first-order Markov chain obtained by augmenting the stacking process so that it becomes Markov of order~1.
There are multiple equivalent choices of state (e.g.\ reichweite-$s$ blocks of consecutive layers, or a sufficiently large augmented state containing the needed memory). Denote the state space by $\mathcal{Z}$ with $|\mathcal{Z}|=M$, and write
\[
\Xi \in \mathbb{R}^{M\times M},\qquad
\xi_{ij} = \mathbb{P}(Z_{n+1}=j\mid Z_n=i),
\]
for the transition matrix. Let $\bm\pi$ be the stationary distribution (row vector) satisfying
\[
\bm\pi \,\Xi = \bm\pi,\qquad \sum_{i=1}^M \pi_i = 1.
\]
(These are the standard stationary conditions used in Hart--Hansen--Kuhs. \cite{hart2021acentric})

\subsection{Define the complex ``observable'' attached to a state}
For a given scattering vector $\bm Q$, define a complex amplitude attached to state $i\in\mathcal{Z}$:
\[
\Phi_i(\bm Q)\in\mathbb{C}.
\]
In the \emph{block} formulation of Hart--Hansen--Kuhs, $\Phi_i(\bm Q)$ is the structure factor $F_i(\bm Q)$ of a block of $s$ layers. \cite{hart2021acentric}
In the \emph{layer} formulation used earlier in this document, $\Phi_i(\bm Q)$ can be taken as the single-layer amplitude $g_{X(i)}(\bm Q)$ implied by the ABC letter encoded by state $i$.

Define the (rank-one, Hermitian) \emph{structure matrix}
\begin{equation}
F(\bm Q) \in \mathbb{C}^{M\times M},\qquad
F_{ij}(\bm Q) := \Phi_i(\bm Q)^\ast\,\Phi_j(\bm Q).
\label{eq:structure_matrix_def}
\end{equation}
By construction, $F(\bm Q)$ is Hermitian so $F(\bm Q)^{T}=F(\bm Q)^\ast$.

\subsection{Pair-correlation and the correlation function}
Define the pair-correlation function
\[
G_{ij}(\Delta) := \mathbb{P}(Z_n=i,\ Z_{n+\Delta}=j)
\qquad (\Delta\ge 0).
\]
Stationarity plus the Markov property give the standard identity
\begin{equation}
G_{ij}(\Delta)=\pi_i\,(\Xi^\Delta)_{ij},
\qquad \Delta\ge 0.
\label{eq:Gij_pi_Xipower}
\end{equation}
This is the same relation used by Hart--Hansen--Kuhs (stated in their Appendix~B in the context of $G_{ij}(m_3)$). \cite{hart2021acentric}

Now define the correlation function (compare to our earlier definition $C_\Delta(\bm Q)=\mathbb{E}[g_{X_n}(\bm Q)^\ast g_{X_{n+\Delta}}(\bm Q)]$):
\begin{equation}
C_\Delta(\bm Q)
:=\mathbb{E}\big[\Phi_{Z_n}(\bm Q)^\ast\,\Phi_{Z_{n+\Delta}}(\bm Q)\big],
\qquad \Delta\ge 0.
\label{eq:Cdelta_def}
\end{equation}
Expanding \eqref{eq:Cdelta_def} over the finite state space and using \eqref{eq:Gij_pi_Xipower} yields
\begin{align}
C_\Delta(\bm Q)
&= \sum_{i=1}^M\sum_{j=1}^M
\mathbb{P}(Z_n=i,\ Z_{n+\Delta}=j)\,
\Phi_i(\bm Q)\,\Phi_j(\bm Q)^\ast
\nonumber\\
&= \sum_{i=1}^M\sum_{j=1}^M
 \pi_i\,(\Xi^\Delta)_{ij}\,F_{ij}(\bm Q).
\label{eq:Cdelta_double_sum}
\end{align}
This is the same algebraic object as Hart--Hansen--Kuhs' $Y_{m_3}(\bm Q)$ (their Eq.~(9)), with $\Delta$ playing the role of their separation index $m_3$. \cite{hart2021acentric}

\subsection{Trace form (compact closed form)}
Using $F(\bm Q)^T=F(\bm Q)^\ast$ and basic trace identities, Hart--Hansen--Kuhs show that the corresponding quantity can be written as a trace. \cite{hart2021acentric}
Applying the same step to \eqref{eq:Cdelta_double_sum} gives
\begin{equation}
C_\Delta(\bm Q)
=\operatorname{Tr}\!\left(F(\bm Q)^\ast\,\operatorname{Diag}(\bm\pi)\,\Xi^\Delta\right),
\qquad \Delta\ge 0.
\label{eq:Cdelta_trace}
\end{equation}
Equation \eqref{eq:Cdelta_trace} is the desired closed form for $C_\Delta(\bm Q)$ in terms of the transition matrix $\Xi$, the stationary distribution $\bm\pi$, and the structure matrix $F(\bm Q)$.
(Compare directly to Hart--Hansen--Kuhs' trace expression for $Y_{m_3}(\bm Q)$. \cite{hart2021acentric})

\subsection{Spectral/Jordan closed forms}
The remaining dependence on $\Delta$ is entirely through $\Xi^\Delta$.

\paragraph{Diagonalizable case.}
If $\Xi$ is diagonalizable, write
\[
\Xi = Q\,\Lambda\,Q^{-1},\qquad
\Lambda=\operatorname{Diag}(\lambda_1,\dots,\lambda_M),
\]
so that
\[
\Xi^\Delta = Q\,\Lambda^\Delta\,Q^{-1},
\qquad
\Lambda^\Delta=\operatorname{Diag}(\lambda_1^\Delta,\dots,\lambda_M^\Delta).
\]
Define similarity-transformed matrices
\[
\widehat{P} := Q^{-1}\operatorname{Diag}(\bm\pi)\,Q,
\qquad
\widehat{F}(\bm Q) := Q^{-1}F(\bm Q)\,Q.
\]
Substituting into \eqref{eq:Cdelta_trace} and using cyclic invariance of the trace gives
\begin{equation}
C_\Delta(\bm Q)
=\operatorname{Tr}\!\left(\widehat{F}(\bm Q)^\ast\,\widehat{P}\,\Lambda^\Delta\right)
=\sum_{n=1}^M \left(\widehat{F}(\bm Q)^\ast\,\widehat{P}\right)_{nn}\,\lambda_n^\Delta.
\label{eq:Cdelta_eigen_sum}
\end{equation}
Thus $C_\Delta(\bm Q)$ is a finite linear combination of exponentials in $\Delta$, with decay rates governed by the eigenvalues of $\Xi$.

\paragraph{Defective (Jordan) case.}
If $\Xi$ is not diagonalizable, Hart--Hansen--Kuhs treat the Jordan form explicitly (see their Appendix~B discussion of Jordan blocks). \cite{hart2021acentric}
Write
\[
\Xi = Q\,J\,Q^{-1},\qquad
J=\operatorname{blockdiag}(J_1,\dots,J_P),
\]
where each Jordan block has the form $J_p=\lambda_p(I+N_p)$ with $N_p$ nilpotent (superdiagonal ones).
Then
\begin{equation}
J_p^\Delta
=\lambda_p^\Delta (I+N_p)^\Delta
=\sum_{r=0}^{d_p-1} \binom{\Delta}{r}\,\lambda_p^{\Delta-r}\,N_p^{r},
\label{eq:Jpower_binomial}
\end{equation}
where $d_p$ is the block size and $N_p^{d_p}=0$.
Hence $\Xi^\Delta = Q\,J^\Delta Q^{-1}$ and $C_\Delta(\bm Q)$ becomes a finite sum of terms of the form
\[
\binom{\Delta}{r}\,\lambda_p^{\Delta-r}
\quad\text{(polynomial-in-$\Delta$ times exponential).}
\]
This provides a fully analytic closed form for $C_\Delta(\bm Q)$ even in the defective case.

\subsection{Remarks}
\begin{itemize}
\item Equations \eqref{eq:Cdelta_double_sum}--\eqref{eq:Cdelta_trace} are the key identities: once the stacking model is expressed as a finite-state Markov chain with transition matrix $\Xi$, the correlation function $C_\Delta(\bm Q)$ is obtained by a single matrix power $\Xi^\Delta$ and a bilinear/trace contraction with $F(\bm Q)$ and $\bm\pi$.
\item In the full cross-section, one typically requires not only $C_\Delta$ for fixed $\Delta$, but weighted sums over $\Delta$ (e.g.\ $\sum_{\Delta}(N_c-|\Delta|)C_\Delta e^{i\Delta\phi}$). Hart--Hansen--Kuhs provide closed forms for these weighted sums using eigenvalues/Jordan blocks; the same spectral machinery extends immediately once \eqref{eq:Cdelta_trace} is established. \cite{hart2021acentric}
\end{itemize}

\subsection*{C. Make the reichweite-$4$ disorder computable via state augmentation}
The key technical step is to represent the induced process $(X_n)$ (or a sufficient statistic for $C_\Delta$) as a \emph{first-order} Markov chain on a \emph{finite} state space.

Because $S_n$ is second-order Markov in \eqref{eq:abcd_def}, one natural augmentation is to define pair-states
\[
U_n := (S_{n-1},S_{n}) \in \{HH,HK,KH,KK\}.
\]
Then $(U_n)$ is first-order Markov with a $4\times4$ transition matrix determined by $(\alpha,\beta,\gamma,\delta)$:
from each pair, the next symbol is $K$ with probability $\alpha,\beta,\gamma,\delta$ respectively, and $H$ otherwise, and the pair window shifts.

However, $C_\Delta$ depends on \emph{ABC labels} $X_n$, not only on HK symbols. To connect them, one introduces an auxiliary binary ``step'' variable $\sigma_n\in\{\pm1\}$ encoding whether the transition $X_n\to X_{n+1}$ is cyclic or anti-cyclic (Appendix~A). Then:
\begin{itemize}
\item $S_n$ tells whether $\sigma_{n}=\sigma_{n-1}$ (cubic, $K$) or $\sigma_{n}=-\sigma_{n-1}$ (hexagonal, $H$).
\item Given an initial $(X_0,\sigma_0)$, the entire ABC sequence is determined by the step sequence.
\end{itemize}
Therefore one may build a finite state
\[
Z_n := (X_n,\sigma_{n-1},U_n),
\]
or any equivalent finite-memory representation, turning the stacking into a first-order Markov chain with transition matrix $T(\alpha,\beta,\gamma,\delta)$.

\paragraph{Computing correlations.}
Let the state space be $\mathcal{Z}$ and let $\pi$ be the stationary distribution of $Z_n$ (solve $\pi^\top T=\pi^\top$, $\sum\pi=1$). Define a vector-valued observable
\[
\varphi(z;\bm Q)=g_{X(z)}(\bm Q),
\]
i.e.\ the layer amplitude associated to the ABC letter in state $z$.
Then the correlation \eqref{eq:Cdef} can be written as a matrix expression:
\begin{equation}
C_\Delta(\bm Q)
=
\sum_{z,z'\in\mathcal{Z}} \pi(z)\, \varphi(z;\bm Q)\,
[T^\Delta]_{z z'}\,\varphi(z';\bm Q)^\ast,
\qquad \Delta\ge 0.
\label{eq:Cmatrix}
\end{equation}
This is the mathematically essential point: the forward model reduces to powers of a fixed matrix $T$ determined by $(\alpha,\beta,\gamma,\delta)$.

\paragraph{Efficiency (important for inverse fitting).}
Naively computing $T^\Delta$ for many $\Delta$ is expensive. Standard accelerations include:
\begin{itemize}
\item diagonalization / Schur decomposition of $T$ once per parameter set, allowing fast evaluation of $T^\Delta$-weighted sums;
\item rewriting the weighted sum over $\Delta$ in \eqref{eq:master_corr} as a closed-form series in eigenvalues (geometric-series identities), with careful handling of near-unit eigenvalues;
\item using linear recurrences when the state space is small (e.g.\ compute $C_{\Delta+1}$ from $C_\Delta$ by propagating a vector through $T$).
\end{itemize}
Any mathematically equivalent technique is acceptable.

\subsection*{D. Powder averaging as a numerically challenging sphere integral}
For each $Q_i$, the powder average \eqref{eq:powder} requires evaluating $\mathbb{E}[|\psi(Q_i\hat{\bm u})|^2]$ for many directions $\hat{\bm u}\in S^2$.

\paragraph{Why it is hard.}
Finite-size lattice sums produce sharp peaks when components of $\bm Q$ align with reciprocal-lattice directions (Dirichlet-kernel behavior). As $N_a,N_b,N_c$ grow, these peaks become narrow, so the integrand is almost zero over most of $S^2$ and very large in small neighborhoods.

\paragraph{Stable and robust strategies.}
A successful approach typically combines:
\begin{itemize}
\item a good spherical quadrature rule (or randomized quasi-Monte Carlo),
\item adaptivity: increase angular resolution until convergence, or refine locally near detected peaks,
\item numerically stable evaluation of lattice-sum factors near their removable singularities (use $\mathrm{sinc}$-style expansions when denominators are small).
\end{itemize}
The benchmark can be made intentionally sensitive to these issues by choosing $(N_a,N_b,N_c)$ that yield thin features.

\subsection*{E. Lorentzian convolution in scalar $Q$}
After computing $I(Q_i)$, apply \eqref{eq:lorentz_conv}. On a discrete $Q$-grid, this becomes a weighted sum approximating an integral. If the grid is uniform, FFT-based convolution is natural; otherwise, one uses direct quadrature, possibly with truncation because $R_\Gamma(\Delta Q)$ decays as $1/\Delta Q^2$.

\subsection*{F. Inverse problem: estimating $(\alpha,\beta,\gamma,\delta,\Gamma)$}
Let $\theta=(\alpha,\beta,\gamma,\delta,\Gamma)$
A standard estimator is constrained nonlinear least squares (or a Poisson likelihood if counts are modeled):
\[
\min_{\theta}\ \sum_{i=1}^M \big(I_i - \,I_{\mathrm{obs}}(Q_i;\theta)\big)^2
\quad\text{s.t.}\quad
\alpha,\beta,\gamma,\delta\in[0,1],\ \Gamma>0.
\]
Practical techniques:
\begin{itemize}
\item constraint handling via reparameterization (logistic for $\alpha,\beta,\gamma,\delta$; log for $\Gamma$);
\item multi-start or coarse-to-fine optimisation (important due to nonconvexity and expensive forward evaluations);
\item alternating updates: since $\Gamma$ is 1D, update $\Gamma$ by line search while holding stacking parameters fixed, then update stacking parameters, and repeat;
\item progressive accuracy: begin optimisation with a cheaper powder integration tolerance, tighten near the optimum.
\end{itemize}

\bigskip
\hrule
\bigskip

\subsection*{Appendix A: Constructive link between HK symbols and ABC stacking (optional)}
One convenient representation introduces a binary step sequence $\sigma_n\in\{\pm1\}$, where $\sigma_n=+1$ means the transition $X_n\to X_{n+1}$ is cyclic (A$\to$B$\to$C$\to$A) and $\sigma_n=-1$ is anti-cyclic (A$\to$C$\to$B$\to$A). Given $X_0$ and $\sigma_0$, the entire ABC sequence is determined.

The Jagodzinski symbol $S_n$ (for the middle layer of $X_{n-1},X_n,X_{n+1}$) satisfies:
\[
S_n = K \iff \sigma_n=\sigma_{n-1},
\qquad
S_n = H \iff \sigma_n=-\sigma_{n-1}.
\]
Thus, given $\sigma_0$ and the HK sequence $(S_n)$, one can recover $\sigma_n$ recursively, and then recover the ABC labels $(X_n)$.

This appendix is \emph{not} needed if you compute correlations directly via a finite-state Markov model; it is provided only to make the induced ABC process unambiguous and to enable optional Monte Carlo checks.

\section{Plausible geometric inputs from the literature (to provide to the agent)}

This section collects concrete, literature-backed values (and simple conventions) for:
(i) the in-plane offsets corresponding to ABC stacking of close-packed layers, and
(ii) atomic positions (at least for the oxygen sublattice) for cubic ice (Ice Ic) and hexagonal ice (Ice Ih),
suitable for constructing synthetic neutron diffraction data.

\subsection{In-plane ABC offsets for close-packed layers}
A standard convention for the lateral offsets of close-packed layers uses a 2D triangular lattice in each layer.
Let the in-plane lattice vectors be
\[
\mathbf{a}_1=(a_\parallel,0,0),\qquad
\mathbf{a}_2=\Big(\tfrac{a_\parallel}{2},\tfrac{\sqrt{3}a_\parallel}{2},0\Big).
\]
Define the three possible layer offsets (A, B, C) by
\[
\mathbf{s}_A=\mathbf{0},\qquad
\mathbf{s}_B=\frac{2\mathbf{a}_1+\mathbf{a}_2}{3},\qquad
\mathbf{s}_C=\frac{\mathbf{a}_1+2\mathbf{a}_2}{3}.
\]
Equivalently, in fractional coordinates with respect to $(\mathbf{a}_1,\mathbf{a}_2)$,
\[
\mathbf{s}_A=(0,0),\quad
\mathbf{s}_B=\Big(\tfrac{2}{3},\tfrac{1}{3}\Big),\quad
\mathbf{s}_C=\Big(\tfrac{1}{3},\tfrac{2}{3}\Big),
\]
up to relabeling $B\leftrightarrow C$ depending on the origin/handedness convention. This is the canonical ABC shifting used to represent
\emph{hexagonal} stacking as $\ldots ABAB\ldots$ and \emph{cubic} stacking as $\ldots ABCABC\ldots$ along the stacking axis.

\subsection{Cubic ice (Ice Ic): lattice and oxygen positions}
A convenient crystallographic entry for Ice Ic is the Crystallography Open Database (COD) structure
\textbf{COD 1541503} (Lisgarten \& Blackman, \emph{Nature} 1956), space group $Fd\bar 3 m$,
with conventional cubic cell parameter
\[
a=b=c=6.351~\text{\AA}.
\]
The CIF provides oxygen as a symmetry site at fractional coordinates $(0,0,0)$ (with the full oxygen network generated by the space group).
The corresponding COD record is available at:
\[
\texttt{https://www.crystallography.net/cod/1541503.cif}.
\]
In explicit fractional coordinates (conventional cubic cell), an equivalent description of the oxygen sublattice
(the diamond-type network on an fcc lattice with a $(\tfrac14,\tfrac14,\tfrac14)$ basis shift) uses the 8 oxygen positions
\[
\begin{aligned}
& (0,0,0),\ (0,\tfrac12,\tfrac12),\ (\tfrac12,0,\tfrac12),\ (\tfrac12,\tfrac12,0),\\
& (\tfrac14,\tfrac14,\tfrac14),\ (\tfrac14,\tfrac34,\tfrac34),\ (\tfrac34,\tfrac14,\tfrac34),\ (\tfrac34,\tfrac34,\tfrac14).
\end{aligned}
\]
These positions are often sufficient for a benchmark forward model if one adopts an \emph{oxygen-only} scattering model
(or a molecular form factor centered on O), which also avoids the additional complexity of hydrogen disorder.

\subsection{Hexagonal ice (Ice Ih): lattice and sources for atomic coordinates}
For Ice Ih, COD provides multiple vetted entries in space group $P6_3/mmc$.
Two useful COD records (with low-temperature lattice parameters) are:
\begin{itemize}
\item \textbf{COD 9015208}: $P6_3/mmc$ with reported parameters
\[
a=b=4.497479~\text{\AA},\qquad c=7.322382~\text{\AA},
\]
visible via COD search results.%
\footnote{\texttt{https://qiserver.ugr.es/cod/result.php?...} (search results listing including COD 9015208).}
\item \textbf{COD 1572235}: an IUCrJ 2024 entry (electron diffraction) with reported parameters
\[
a=b=4.3815~\text{\AA},\qquad c=7.183~\text{\AA}\quad (\sim 81~\text{K}),
\]
available at:
\[
\texttt{https://www.crystallography.net/cod/1572235.html}.
\]
\end{itemize}
\noindent
\textbf{Recommendation for the benchmark:} bundle the chosen Ice Ih CIF directly with the task inputs
so that the agent can parse the oxygen coordinates (and, if desired, hydrogen/deuterium sites) without ambiguity
about origin settings or symmetry conventions. This keeps the forward model deterministic and avoids
errors due to differing crystallographic conventions.

\subsection{How these inputs connect to the stacking-disorder construction}
A benchmark-friendly construction that matches the ABC stacking model is:
\begin{enumerate}
\item Choose a \emph{reference layer} consisting of a 2D set of atomic sites (e.g.\ oxygen positions) in the basal plane.
\item Generate a stacking sequence $(X_n)_{n=0}^{N_c-1}$ with $X_n\in\{A,B,C\}$ under the close-packed constraint,
using the Reichweite-4 Markov model (via HK notation) with parameters $(\alpha,\beta,\gamma,\delta)$.
\item Place layer $n$ at height $z_n = n\,d$ and lateral offset $\mathbf{s}_{X_n}\in\{\mathbf{s}_A,\mathbf{s}_B,\mathbf{s}_C\}$,
producing atom positions of the form
\[
\mathbf{r} = m_1\mathbf{a}_1 + m_2\mathbf{a}_2 + \mathbf{s}_{X_n} + z_n\,\hat{\mathbf{z}} + \mathbf{s}_p,
\]
where $\mathbf{s}_p$ are the in-layer basis coordinates, and $(m_1,m_2)$ index the finite in-plane extent.
\item (Optional) If including H/D explicitly, add their positions relative to each oxygen according to a chosen average/disorder model,
or supply H/D sites from a CIF-based structural model.
\end{enumerate}

\subsection{Citations / links (for the document)}
\begin{itemize}
\item Ice Ic CIF: COD 1541503 \texttt{https://www.crystallography.net/cod/1541503.cif}.
\item Ice Ih entry example: COD 1572235 \texttt{https://www.crystallography.net/cod/1572235.html}.
\item Ice Ih search listing (includes COD 9015208): \texttt{https://qiserver.ugr.es/cod/result.php?...}.
\end{itemize}

\bibliographystyle{plain}
\begin{thebibliography}{1}
\bibitem{hart2021acentric}
A.~G. Hart, T.~C. Hansen, and W.~F. Kuhs.
\newblock Acentricity in aperiodic crystals.
\newblock \emph{arXiv:2102.08418}, 2021.
\end{thebibliography}

\end{document}
